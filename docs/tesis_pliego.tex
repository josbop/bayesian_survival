\documentclass[10pt,a4paper]{article}
\usepackage[utf8]{inputenc}
\usepackage[spanish]{babel}
\usepackage{amsmath}
\usepackage{amsfonts}
\usepackage{amssymb}
\author{José Pliego San Martín}
\title{Tesis}
\begin{document}

\maketitle

\setlength{\parindent}{0pt}

\section{Simulación de variables aleatorias}

La simulación de variables aleatorias se basa en la posibilidad de simular observaciones de una distribución uniforme en el intervalo $(0, 1)$. Estas observaciones se pueden utilizar para simular distribuciones específicas por medio de métodos como el de la transformada inversa o métodos de aceptación y rechazo. En este capítulo se presentan la construcción y las principales características de dichos métodos.

\subsection{Método de la transformada inversa}

Recordemos que la función de distribución $F$ de una variable aleatoria $X$
\begin{align*}
F: \mathbb{R}&\to [0,1]\\
F(x) &= P\{X \leq x\}\\
\end{align*}
tiene las características siguientes:

\begin{enumerate}
\item $F$ es no decreciente,\\
\item $F$ es continua por la derecha,\\
\item $\lim\limits_{x\to +\infty}F(x) = 1$ y $\lim\limits_{x\to -\infty}F(x) = 0$.\\
\end{enumerate}

Se define la inversa generalizada de $F$, denotada $F^{-1}:[0,1]\to \mathbb{R}$, como $$F^{-1}(y) = \inf \{x\in \mathbb{R}: F(x) \geq y\}.$$ Con esta definición se puede verificar que

\begin{itemize}
\item $F^{-1}(F(x)) \leq x$ porque $F$ es no decreciente,\\
\item $F(F^{-1}(y)) \geq y$ por la definición de $F^{-1}$ y\\
\item $F^{-1}$ es no decreciente.\\
\end{itemize}

Más aún, si $F$ es estrictamente creciente se tiene que $F^{-1}(F(x)) = x$ y si $F$ es continua se cumple $F(F^{-1}(y)) = y$, por lo que $F^{-1}$ define la inversa de $F$ en el sentido usual si $F$ es continua y monótona estricta.\\

\textbf{El método de la transformada inversa.} Sea $F$ una función de distribución, $F^{-1}$ su inversa generalizada y sea $U\sim U(0,1)$. Entonces la variable aleatoria $X$ definida como $X = F^{-1}(U)$ tiene función de distribución $F$.\\

\textit{Demostración.} Sea $x\in \mathbb{R}$ fija. Se quiere probar la contención de los eventos $$\{U<F(x)\}\subseteq\{F^{-1}(U)\leq x\}\subseteq\{U\leq F(x)\}.$$ Sea $u$ una observación de la variable aleatoria $U$.

$$u<F(x) \implies F^{-1}(u) \leq F^{-1}(F(x)) \leq x$$
$\therefore \hspace{2pt} \{U<F(x)\} \subseteq\{F^{-1}(U)\leq x\}$ y


$$F^{-1}(u)\leq x \implies u\leq F(F^{-1}(u))\leq F(x)$$
$\therefore \hspace{2pt} \{F^{-1}(U)\leq x\}\subseteq \{U \leq F(x)\}$.\\

De esta forma, se tiene que 
\begin{align*}
P\{U\leq F(x)\} = P\{U<F(x)\} &\leq P\{F^{-1}(U)\leq x\} \leq P\{U \leq F(x)\}\\
\implies F(x) = P\{U \leq F(x)\} &= P\{F^{-1}(U)\leq x\} = P\{X\leq x\}\\
\implies P\{X \leq x\} &= F(x),\\
\end{align*}

con lo que se concluye que $F$ es la función de distribución de $X$.


\end{document}